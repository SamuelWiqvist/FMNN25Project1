\documentclass[a4paper]{article}
\usepackage[utf8]{inputenc}
\usepackage{amsmath}
\usepackage{amssymb}
\usepackage{graphicx}
\newcommand{\ddd}{\mathrm{d}}
\newcommand{\dd}{\mathrm{\,d}}
\newcommand{\ddn}{\mathrm{d}}
\newcommand{\ee}{\mathrm{e}}
\newcommand{\ii}{\mathrm{i}}
\newcommand{\rr}{\mathbb{R}}
\newcommand{\cc}{\mathbb{C}}
\newcommand{\zz}{\mathbb{Z}}
\newcommand{\qq}{\mathbb{Q}}

\title{Project 1 - De Boor's Algorithm}
\author{Linus Jangland, Anton Roth & Samuel Wiqvist}

\begin{document}
\section{Design class Spline}
The attributes of our class are the control points (de Boor points), which describes the design of the curve that is to be recreated, and the grid points $u$. In the class Spline there are three "public" functions that can be called. The class' call function evaluates the value of the spline corresponding to a certain grid value $u$. This function is based on several other hidden class functions. Another "public" function is a plot function which plots the spline for all the grid points and in the plot are also the control points drawn. The last unhidden function returns the basis function $N_i^3$ for an input $i$. This function is based on the definition of the basis functions and thus generates the $N_i^3$ with a recursion algorithm. This means that this function is not based upon the already created 

Tests have been implemented to the Spline class.  
\end{document}